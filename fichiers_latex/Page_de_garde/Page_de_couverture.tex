\Sethpageshiftt{11mm}   
\Setvpageshiftt{13mm}   
\vspace{0mm}

\Resume{
La maîtrise de la qualité de la production est un objectif particulièrement important pour la croissance des industries.
Maîtriser la qualité d'un produit nécessite de mesurer.
% Il est possible de mesurer des variables sur le procédé, des caractéristiques ou sur le produit fini.
Cependant les contraintes industrielles, comme la durée du cycle de production, ont limité le déploiement de la mesure des caractéristiques des produits directement au sein des lignes de production.
Afin de répondre à ces contraintes dans le cadre du procédé d'injection-moulage des thermoplastiques, notre approche utilise des moyens de mesure non invasifs sur des pièces qui viennent de sortir de la machine.
De plus, les cahiers des charges des produits sont de plus en plus exigeants.
Ils spécifient des caractéristiques géométriques et des caractéristiques d'aspect qui sont difficiles à formaliser.
Le contrôle automatisé de ces caractéristiques est un enjeu crucial pour maitriser la production.
Nous étudions l'apport de l'imagerie non-conventionnelle, la thermographie et la polarimétrie, pour mettre en valeur les défauts des pièces.
Notre approche utilise l'apprentissage statistique pour que l'expert humain transmette sont savoir concernant la notion de que qualité d'une pièce.
À l'aide d'algorithmes de Deep Learning, notre système de contrôle apprend une métrique de la qualité d'une pièce, à partir des données brutes des capteurs.
Afin de répondre à l'ensemble des contraintes, nous avons conçu un système de contrôle adapté qui intègre les capteurs, l'éclairage et le système embarqué qui réalise les traitements de la mesure, au plus près des capteurs.
% Ce système est rapidement installé, mobile.
Plusieurs cas d'études industriels permettent d'évaluer les performances et la viabilité de la solution retenue.
}

\Motscles{Qualité, Métrologie, Thermographie, Polarimétrie, Apprentissage de métrique, Deep Learning}

\Abstr{
Online control of the product quality is an important objective for industries growth.
Quality control requires measurements.
% It is possible to measure variables on the process, characteristics or the finished product.
However, industrial constraints, such as the duration of the production cycle, have limited the deployment of measurement of product characteristics, within production lines.
In the context of the thermoplastic injection-molding process, our approach uses non-invasive measuring methods on parts that have just come out of the machine.
Furthermore, specifications on products are more and more tighter.
Specifications include geometric features, as well as aspect features, which are difficult to formalize.
The automated control of these characteristics is a crucial stake to control the production.
We study the contribution of non-conventional imaging, thermography and polarimetry, to highlight the defects of the pieces.
Our approach uses statistical learning for the human expert to transfer his knowledges about the notion of parts quality.
Using Deep Learning algorithms, our control system learns a metric of parts quality, from the sensors raw data.
In order to meet all industrial constraints, we have designed a suitable control system that integrates sensors, lighting and the embedded measurements processing as closer to sensors.
% This system is quickly installed, mobile.
Several cases of industrial studies make it possible to evaluate the performance and the viability of the chosen solution.
}

\Keywords{Quality, Metrology, Thermography, Polarimetry, Metric learning, Deep Learning}

\MakeUGthesePDC    

