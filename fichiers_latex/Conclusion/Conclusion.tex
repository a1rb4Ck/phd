%%%%%%%%%%%%%%%%%%%%%%%%%%%%%%%%%%%%%%%%%%%%%%%%%%%%%%%%%%%%%%%%%%%%%%%%%%

%%%%%                           Conclusion Générale                 %%%%%%
%%%%%%%%%%%%%%%%%%%%%%%%%%%%%%%%%%%%%%%%%%%%%%%%%%%%%%%%%%%%%%%%%%%%%%%%%%

\phantomsection 
\addcontentsline{toc}{chapter}{Conclusion générale}
\addtocontents{toc}{\protect\addvspace{10pt}}

\vspace*{-1cm}
\begin{flushright}
\section*{\fontsize{20pt}{20pt}\selectfont\textnormal{Conclusion générale}}
\end{flushright}
\vspace{2cm}

\lhead[\fancyplain{}{Conclusion générale}]
      {\fancyplain{}{}}
\chead[\fancyplain{}{}]
      {\fancyplain{}{}}
\rhead[\fancyplain{}{}] 
      {\fancyplain{}{Conclusion générale}}
\lfoot[\fancyplain{}{}]
      {\fancyplain{}{}}
\cfoot[\fancyplain{}{\thepage}]
      {\fancyplain{}{\thepage}}
\rfoot[\fancyplain{}{}]
     {\fancyplain{}{\scriptsize}}

%%%%%%%%%%%%%%%%%%%%%%%%%%%%%%%%%%%%%%%%%%%%%%%%%%%%%%%%%%%%%%%%%%%%%%%%%%
%%%%%                      Start part here                          %%%%%%
%%%%%%%%%%%%%%%%%%%%%%%%%%%%%%%%%%%%%%%%%%%%%%%%%%%%%%%%%%%%%%%%%%%%%%%%%%
\label{ch:conclusion}
\lettrine[lines=1]{L}{ }'industrie manufacturière cherche à produire en grande quantité des pièces à la qualité optimale.
La qualité d'une pièce est l'ensemble de ses caractéristiques qui répondent à un besoin.
Les besoins spécifiées dans les cahiers des charges concernent la fonction des pièces, mais aussi leur aspect.
En particulier, l'aspect visuel devient, pour le client, un indicateur de la qualité fonctionnelle du produit.
Contrôler la qualité d'un produit nécessite ainsi de mesurer ses caractéristiques fonctionnelles et ses caractéristiques d'aspect.
Notre travail propose d'utiliser des techniques d'imagerie non-conventionnelle pour contrôler l'aspect et la géométrie.

% De plus, dans les cahiers des charges, le niveau d'exigence en matière de production est croissant.
De plus, l'automatisation des procédés de fabrication demande de pouvoir écarter les pièces non-conformes de la chaîne de production.
En particulier, les assemblages automatisés de plusieurs pièces élémentaires augmentent singulièrement la sensibilité aux pièces non-conformes.
En effet, la présence de produits non conformes, dans un processus automatisé, peut conduire à des arrêts de chaîne de production coûteux.
Dans le cadre de l'injection-moulage des thermoplastiques, les secteurs automobile et médical sont pionniers de cette automatisation.
Le taux de pièces non-conformes acceptable est particulièrement faible ; de l'ordre de quelques pièces par millions dans les lots livrés.
Aujourd'hui, les niveaux de qualité inférieurs à cinq pièces non-conformes par million ne peuvent être garantis que par le contrôle à cent pourcent des pièces et l'élimination des non-conformes.  % (5 ppm)

% Il est alors indispensable de réaliser le contrôle de la qualité de cent pourcent des pièces.
Cependant, lorsqu'il est réalisé par un opérateur humain, le coût économique du contrôle de la qualité à cent pourcent n'est pas viable pour un volume de production industriel.
Le contrôle qualité ne doit pas non plus interrompre le flux de production.
C'est pourquoi l'automatisation du contrôle de la qualité est obligatoire.
Un des objectif de notre travail est de réaliser le contrôle de la qualité de cent pourcent des pièces directement en ligne de production.

% Les coûts de non-qualité sont engendrés par la fabrication de produits non-conformes.
De plus, il est coûteux d'apporter de la valeur ajoutée par des opérations de finition à un produit qui pourrait être déclaré non conforme dès l'étape qui a causée la non-conformité.
Dans le cadre de l'injection-moulage des thermoplastiques, le contrôle de la qualité des produits dès la sortie du moule permet d'éliminer ce coût de non-qualité.
La gestion des produits non-conforme est également simplifiée.  % La gestion du cycle de vie 
Si elle est contrôlée à temps, une pièce non-conforme pourrait être recyclée dès sa production.
De plus, un circuit spécifique de recyclage, adapté au matériau en présence, pourrait-être mis en place à la suite de chaque étape de contrôle de la qualité, tout au long de la production.

Dans une démarche de maîtrise du procédé de fabrication, le contrôle de la qualité en ligne pourrait permettre de détecter la dérive du procédé au plus tôt.
Il deviendrait alors possible d'alerter ou de stopper le moyen de production, pour ne plus produire de pièces non-conformes.
Cela permettrait d'économiser la matière première de plusieurs centaines, voir de plusieurs milliers de pièces.
Si ces pièces poursuivent leur chemin dans des étapes de finitions, le coût de la finition est également économisé.
Dans notre cadre d'étude, le procédé d'injection-moulage des thermoplastiques est généralement stable et la dérive du procédé apparait rarement.
Un opérateur humain peut généralement superviser plusieurs machines en parallèle.
Il est plus courant de devoir écarter des pièces non-conformes dûs à des causes exceptionnelles (défaillance du procédé), qu'à une dérive du procédé.
Dans le cas d'une défaillance exceptionnelle, le contrôle en ligne permet une réactivité immédiate pour réparer et ne pas impacter l'ensemble de la chaîne de fabrication.
Dans le cas d'une dérive du procédé, le contrôle en ligne permet d'alerter, voir d'interrompre la production.

% Enfin, le contrôle qualité est une opération indispensable pour respecter le cahier des charges.
% Cependant, cette opération n'ajoute pas de valeur au produit final.
Le contrôle qualité n'ajoute pas de valeur au produit final.
Aussi, le coût acceptable de l'installation d'un système de contrôle automatique est généralement faible.
Seules les productions hautement spécialisées investissent dans son automatisation.
Le coût du système de contrôle doit être inférieur au coût de la production de pièces non-conformes.
Un objectif important de notre travail est de proposer un dispositif de contrôle de la qualité peu coûteux, afin qu'il puisse être largement adopté dans l'industrie de la plasturgie.  % toute l'industrie

%Des contraintes technologiques entrent en jeu dans la conception d'un système prévu pour être intégré en ligne de production.
%En premier lieu, la durée du cycle du procédé de production doit être respecté.
%Le temps de cycle ne doit pas être allongé par la durée de la mesure.
%De plus, les lignes de production sont désormais en constante évolution.
%Un moyen de mesure mobile permettrait d'être positionner au grès des besoins de contrôle.
%Enfin, la notion de qualité d'une pièce doit être spécifiée au sein du système de mesure afin que le système puisse déterminer si la pièce est conforme ou non-conforme.
%Pour répondre à ces contraintes, notre travail propose d'évaluer l'intérêt de l'imagerie non-conventionnelle des pièces dès la sortie du moule.
%Notre approche du traitement des données issues de capteurs s'appuient sur l'apprentissage statistique.
%L'expert humain transmet sa connaissance de la qualité de la pièce au système. % à notre système.
%Il annote les pièces qui sont produites au fur et à mesure du réglage de la presse à injecter.
%Le système apprend alors à associer les pièces conformes ou non-conformes avec les mesures brutes des capteurs.

Enfin, la conception d'un nouveau système de mesure s'inscrit dans une démarche de validation expérimentale de l'ensemble de nos propositions.
Les modèles construits par apprentissage statistique nécessitent un jeu de données composé de nombreux individus.
Posséder un dispositif de mesure nous a permis d'enregistrer les données requises pour l'apprentissage de nos modèles.
Le dispositif de mesure est conçu par intégration verticale : du capteur au traitement des données.
Cela nous a permis de maîtriser l'ensemble de la chaîne d'acquisition et d'analyse des données, et de pouvoir l'optimiser pour répondre à notre problème.
La validation du système a été réalisée sur des presses à injecter industrielles lors de deux essais expérimentaux.
% Des essais expérimentaux réalisés avec un prototype de dispositif de mesures montrent des résultats encourageants.
% TODO: Contributions avant Organisation?

Nous avons proposé un dispositif de mesure de la qualité qui utilise l’apprentissage supervisé.
Le prototype a été évalué en conditions réelles sur des jeux de peu de données.
Néanmoins, les résultats sont encourageants.
La méthode proposée permet de résoudre les problèmes qui limitent actuellement le déploiement du contrôle à cent pour cent en injection plastique.
De futures essais permettront de confronter le dispositif à des quantités de pièces produites importantes.

Actuellement, notre système s’appuie sur une phase d’apprentissage spécifique à chaque pièce, lors de la mise en production d’une pièce.
Il sera intéressant de concevoir un système capable de généraliser la notion de qualité et de généraliser la classification des défauts, pour toutes pièces.

\bigskip

\bigskip

\paragraph{Valorisation scientifiques}\mbox{} \\
%Ce travail s'intéresse à l'apport de l'imagerie non-conventionnelle, aux méthodes d'apprentissages automatiques et à la conception d'un système complet de contrôle automatique de la qualité.
%Ce travail de doctorat proposent une large étude des techniques disponibles qui sont requises pour mettre en place le contrôle qualité des produits par apprentissage sur les chaînes de production.
%
%Ces travaux ont permis le développement d'un système de contrôle automatique de la qualité des pièces plastiques, dès la sortie du moule, de manière non-invasive, sans contact avec la pièce et avec un impact minimal sur le procédé.
%Ce système associe un banc de mesure spécialement développé pour répondre aux contraintes industrielles, ainsi qu'un logiciel afin d'intégrer et de réaliser le traitement des sources d'information.
%Le logiciel utilise les possibilités de calculs distribués offertes par les solutions \textit{Cloud}.
%Une interface homme-machine, sous la forme d'une \textit{application web}, a été développée pour permettre à un expert humain de transmettre sa connaissance à la machine au fur et à mesure de la production de nouvelles pièces.
%Ce système a été évalué en conditions industrielles lors de deux essais, sur le procédé d'injection-moulage des thermoplastiques.
%
%\bigskip

Ces travaux de doctorat ont fait l’objet de communications en congrès scientifiques internationaux \cite{nagorny_towards_2017, nagorny_injection_2017, nagorny_quality_2017, nagorny_generative_2018, nagorny_polarimetric_2019}, ainsi que d'un colloque national \cite{nagorny_towards_2017}.
Un Article de revue à Comité de Lecture a été soumis à la revue \textit{Journal of Electronic Imaging} en Octobre 2019.
Un séminaire de présentation de ces travaux a été présenté au laboratoire SYMME en Décembre 2019.
% Trois posters ont été présentés lors des Journées des Doctorants 2017, 2018, 2019 de l'École Doctorale SISEO de l'Université Savoie Mont Blanc.

\bigskip

\paragraph{Valorisation économique}\mbox{} \\

Notre système de mesure de la qualité a été lauréat du sixième concours \textit{Out Of Labs}\footnote{\href{http://outoflabs.linksium.fr/resultats/}{Résultats de la sixième édition du concours \textit{Out Of Labs}}}, organisé par la \textit{SATT}\footnote{\textit{SATT} : Société d'Accélération du Transfert de Technologies} \href{https://www.linksium.fr/}{Linksium Grenoble Alpes}.
Ce prix nous permet de poursuivre la recherche et les développements, vers une application commerciale.

Le dispositif de contrôle de la qualité conçu pendant ces travaux de doctorat a fait l'objet du dépôt d'un brevet intitulé \citetitle{nagorny_dispositif_2019} \cite{nagorny_dispositif_2019}.
