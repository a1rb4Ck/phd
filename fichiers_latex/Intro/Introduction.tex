%%%%%%%%%%%%%%%%%%%%%%%%%%%%%%%%%%%%%%%%%%%%%%%%%%%%%%%%%%%%%%%%%%%%%%%%%%
%%%%%                        Intro Générale                         %%%%%%
%%%%%%%%%%%%%%%%%%%%%%%%%%%%%%%%%%%%%%%%%%%%%%%%%%%%%%%%%%%%%%%%%%%%%%%%%%
\phantomsection 
\addcontentsline{toc}{chapter}{Introduction}
\addtocontents{toc}{\protect\addvspace{10pt}}

\vspace*{-1cm}
\begin{flushright}
\section*{\fontsize{20pt}{20pt}\selectfont\textnormal{Introduction générale}}
\end{flushright}
\vspace{2cm}

\lhead[\fancyplain{}{Introduction}]
      {\fancyplain{}{}}
\chead[\fancyplain{}{}]
      {\fancyplain{}{}}
\rhead[\fancyplain{}{}]
      {\fancyplain{}{Introduction}}
\lfoot[\fancyplain{}{}]%
      {\fancyplain{}{}}
\cfoot[\fancyplain{}{\thepage}]
      {\fancyplain{}{\thepage}}
\rfoot[\fancyplain{}{}]%
     {\fancyplain{}{\scriptsize}}
     

%%%%%%%%%%%%%%%%%%%%%%%%%%%%%%%%%%%%%%%%%%%%%%%%%%%%%%%%%%%%%%%%%%%%%%%%%%
%%%%%                      Start part here                          %%%%%%
%%%%%%%%%%%%%%%%%%%%%%%%%%%%%%%%%%%%%%%%%%%%%%%%%%%%%%%%%%%%%%%%%%%%%%%%%%
\lettrine[lines=1]{L}{ }'industrie manufacturière est en quête croissante d'une production à la qualité optimale.
Le besoin du client final est la fonction du produit.
Des besoins fonctionnels techniques, comme des besoins fonctionnels d'aspects, apparaissent aujourd'hui sur les mêmes pièces, avec des niveaux d'exigences croissants.
La notion de qualité d'un produit est compliquée à spécifier.
De nombreuses normes permettent de définir des tolérances sur les produits.
Cependant, les normes ne permettent pas de spécifier toutes les caractéristiques d'un produit.
C'est le cas de la spécification de la qualité d'aspect.
Chaque profession a construit sa défauthèque et chaque entreprise du secteur l'a enrichie.
L'exploitation de ces défauthèques, afin de construire une métrique de la qualité, est un véritable défi industriel.

Pour atteindre la qualité optimale, il est nécessaire de maitriser le procédé de production, voir dans le meilleur des cas, de le réguler par optimisation en boucle fermée.
La maîtrise de la qualité s'obtient après une démarche itérative.
Il nécessaire de mesurer la qualité, ce qui demande la mise en place de système de métrologie couplée à une métrique de la qualité adaptée.
La mesure robuste de la qualité permet de mettre sous contrôle le procédé par des moyens statistiques afin de détecter les dérives dans la qualité produite.
Enfin, il devient possible d'effectuer une rétro-action sur les réglages du procédé afin d'optimiser la qualité produite.

La non-qualité est l'ensemble des coûts qui sont engendrés par la fabrication de produits qui ne répondent pas au cahier des charges (ce sont des produits de "mauvaise" qualité).
Les problématiques environnementales et économiques convergent autour de la nécessité de limiter la non-qualité.
Il est coûteux tant économiquement qu'environnementale de procéder aux étapes successives de finitions du produit alors que celui-ci était de mauvaise qualité dès la première étape.
Écarter les produits ne répondant pas au cahier des charges, dès la fin de l'étape du procédé, permet un gain économique conséquent.
La gestion du cycle de vie des produits de mauvaise qualité est également rendue possible.
Un circuit spécifique de recyclage peut être mis en place à chaque étape de production.
Plus encore, détecter une dérive de qualité produite sur le procédé permet d'alerter et de stopper le moyen de production.
Sur des procédés de productions stables où les opérateurs humains interviennent peu, cela permet d'économiser la matière première de plusieurs centaines, voire de plusieurs milliers de pièces.

Plusieurs contraintes industrielles et technologiques entre en jeux dans la conception d'un système de mesure de la qualité.
En premier lieu, le temps de cycle du procédé industriel doit être respecter. Il ne doit pas être impacté par la technologie de mesure.
La notion de qualité doit être spécifiée afin de réaliser une mesure répétable.
Enfin, le coût économique du moyen de métrologie doit être inférieur au coût de la production de rebuts. 

L'injection-moulage des thermoplastiques est un procédé à haute cadence, peu coûteux car répétable.
Une fois la machine de production réglée, le procédé est souvent très stable.
Il peut produire de manière continue plusieurs milliers de pièces sans intervention humaine.
Le coût de la matière première pour le moulage est faible en comparaison du coût des étapes de finitions suivantes, telles que les étapes de peintures.
Éviter aux pièces de mauvaise qualité d'être peintes est une priorité.
En second temps, un système de rétroaction pourra être déployé sur le procédé, afin de maximiser la qualité produite.
% Ce n'est pas la priorité économique.

Le procédé d'injection-moulage des thermoplastiques consiste à injecter un polymère visqueux dans un outillage massif refroidi ; à maintenir une pression de compactage pendant une durée spécifique ; puis à éjecter la pièce tout en préparant en parallèle la matière nécessaire au cycle suivant.
Il est possible d'ajuster précisément les pressions et les durées d'ouvertures et de fermetures de multiples buses d'injection de manière séquentiel.
Ainsi, l'espace des variables de pilotage est grand et les variables sont continues.

\noindent
La production de pièce par moulage des thermoplastiques suit généralement la séquence suivante :
\begin{enumerate}
\item Injection-moulage de la pièce
\item Stockage tampon des pièces
\item Étapes de peintures
\item Stockage
\item Expédition
\end{enumerate}

Afin de minimiser les coûts d'immobilisation du stockage, on cherche à réduire au maximum le stockage entre les étapes.
Les défauts géométriques et d'aspect apparaissent majoritairement lors des étapes de déplacements des pièces entre les étapes successives de la production.
Pour des pièces techniques au cahier des charges rigoureux, un contrôle qualité avant expédition est donc obligatoire.
Les défauts géométriques et d'aspect sont également produits lors de l'étape initiale d'injection-moulage.
De nombreux défauts dit "d'aspect" sont des défauts géométriques à l'échelle de la centaine de micromètres.
Ceux-ci modifient les réflexions des lumières incidentes et cause des changements de luminosité brusques, qui ne sont pas acceptables pour le cahier des charges du produit fini.
Les étapes de peintures peuvent mettre en valeurs des défauts en introduisant des surfaces spéculaires.


\bigskip
\paragraph{Le FUI SAPRISTI : un projet de recherche collaborative de 2016 à 2019}\mbox{} \\

L’obtention d’une chaîne de production compétitive passe par le développement, l’optimisation et l’automatisation de chacune de ces étapes.
Dans cette optique, différents consortiums constitués d’industriels et d’académiques travaillent sur des projets de recherche collaboratives (FUI).
Ces travaux s'inscrivent dans le projet collaboratif FUI\footnote{FUI : Fond Unique Interministériel} SAPRISTI\footnote{SAPRISTI : Système Autocorrectif pour la PRoduction zéro défaut de pIèces pLaSTIques} du \textit{21\up{ème} appel à projet FUI Régions.}\footnote{\href{https://competitivite.gouv.fr/les-appels-a-projets-de-r-d-fui/le-21e-appel-a-projets/les-resultats-du-21e-appel-a-projets-787.html}{Résultats du 21\up{ème} appel à projets FUI sur competitivite.gouv.fr} : \textit{60 projets bénéficieront d’une aide de l’État de 43 Millions d'Euros ainsi que d’une aide des collectivités territoriales et des fonds communautaires (FEDER) de 37,8 Millions d'Euros.}}.
Le projet FUI SAPRISTI est labellisé par trois pôles de compétitivités :
\begin{itemize}
	\item \href{http://www.plastipolis.fr/}{Plastipolis},
	\item \href{https://www.id4car.org}{iDforCAR},
	\item \href{https://www.montblancindustries.com/}{Mont-Blanc Industries}.
\end{itemize}
\noindent
Les partenaires du consortium FUI SAPRISTI sont :
\begin{itemize}
\item \href{http://www.acsysteme.com/}{Acsystème}, développeur et intégrateur de solutions en automatique et traitement de l'information,
\item Le \href{https://ct-ipc.com/}{Centre Technique CT-IPC}, acteur européen en développement et recherche pour le secteur de la plasturgie,
\item Le \href{https://lamcos.insa-lyon.fr/}{laboratoire LaMCoS} de l'\href{https://www.insa-lyon.fr/}{INSA de Lyon}, expert de la modélisation du procédé d'injection-moulage des thermoplastiques,
\item \href{https://www.plasticomnium.com/}{Plastic Omnium}, équipementier automobile mondial
\item \href{https://www.renault.fr/}{Renault}, automobiliste concepteur et équipementier de ses équipements,
\item \href{http://www.sise-plastics.com/}{SISE}, acteur européen de la régulation des procédés en plasturgie, dont notamment le contrôle de l'injection séquentiel,
\item Le \href{http://www.symme.univ-smb.fr/}{laboratoire SYMME}\footnote{SYMME : SYstèmes et Matériaux pour la MÉcatronique} de l'\href{https://www.univ-smb.fr/}{Université Savoie Mont Blanc}, expert en réglage automatique et en optimisation de procédés industriels, dans le cadre de son axe de recherche sur la Maîtrise de la Qualité des Produits pour l'Industrie.
\end{itemize}

L'injection-moulage des thermoplastiques est un procédé de production bas coût haut-volume (\textit{Low-Cost, High Volume}) particulièrement prisé des industriels. Le marché de la plasturgie représente en France 3500 entreprises pour 30,2 Milliards de Chiffre d'Affaires en 2018\footnote{Synthèse 2018 du secteur réalisée par la Fédération Française de la Plasturgie.}. Dans ce contexte, l’enjeu du projet FUI SAPRISTI est triple. Le projet a pour triple objectifs de :
\begin{itemize}
	\item Déterminer des indicateurs pertinents à mettre sous contrôle.
	\item Mieux comprendre les phénomènes physiques qui entrent en jeu dans le procédé, afin de modéliser le procédé en prenant en compte les indicateurs sélectionnés.
	\item Optimiser le procédé en boucle fermé à partir de la modélisation et des indicateurs, afin de produire une qualité optimale.
\end{itemize}

Dans ce cadre, les objectifs pour le laboratoire SYMME sont dans un premier temps de trouver quelles sont les variables pertinentes à étudier sur le procédé.
Ces variables peuvent être des mesures directes sur le procédé ou bien des mesures réalisées sur les produits.
Dans cette démarche, les contraintes industrielles de sont pris en compte pour choisir des moyens de mesures compatibles avec le procédé.
Ensuite, il s'agit de modéliser les effets de ces variables avec la qualité finale des produits.
Pour cela, nous utiliserons des modèles construits par apprentissage statistique.
L'objectif final du FUI SAPRISTI est de mettre sous contrôle le procédé afin de maximiser la qualité produite.
Cet objectif ne peut être atteint que si les précédents objectifs sont résolus.
Cette nécessité de disposer d'un moyen de contrôle automatique de la qualité des produits a donné lieu à notre problématique de recherche.

\bigskip
\paragraph{Problématique de recherche}\mbox{} \\

Dans le cas de l’injection-moulage des thermoplastiques, aucun système de mesure de la qualité ne répond actuellement aux exigences des partenaires du FUI SAPRISTI \ ; que se soient en matière de robustesse de la mesure, ou en matière de compatibilité avec les contraintes posées par le procédé.
Dans ce cadre, ce travail de doctorat cherche à développer un dispositif de mesures non-invasif de la qualité des produits, dès la sortie de la machine, qui répond aux problématiques industrielles.

Nous avons identifié trois verrous au déploiement massif du contrôle qualité à cent pour-cent dès la sortie de la machine :
\begin{itemize}
	\item économique
	\item technologique
	\item humain
\end{itemize}
Nous évaluerons et chercherons à résoudre chacun de ces verrous afin de proposer un moyen de mesure industriellement viable.
Nous présentons les méthodes et les moyens retenus pour lever ces verrous.

Des échanges avec les partenaires du FUI SAPRISTI nous ont permis de prendre un compte les limites que posent le procédé industriel.
La pratique industrielle actuelle propose d'utiliser des mesures intégrées dans l'outillage.
L'intégration des capteurs nécessaire à ces mesures est invasive : l'ingénierie nécessaire à la conception de l'outillage avec capteur est couteuse \ ; l'instrumentation des capteurs est fragile et la maintenance nécessite un arrêt prolongé de la production puisque c'est l'outillage qui doit être expertisé.

La notion de qualité est subjective.
Malgré les normes, la notion de qualité est souvent dépendante des connaissances holistiques du responsable expert qualité.
Il est difficile de modéliser ce savoir dans un système de contrôle automatisé.
Les récentes avancées en apprentissage statistique, dont les réseaux de neurones profonds (\textit{Deep Learning}), permettent de concevoir un système qui intègre un modèle de la notion de qualité.
Ce modèle est construit à partir de l’expertise humaine, par apprentissage supervisé.
L’apprentissage du modèle est réalisé au fur et à mesure de la production, de manière a être intégré au procédé de production.
Ce dispositif de mesure s'appuie également sur l'apport de l'imagerie non-conventionnelle qui permettent d'acquérir des informations que l'œil de l'expert humain ne peut percevoir.
Des essais expérimentaux réalisés avec un prototype de dispositif de mesures montrent des résultats encourageants.

\bigskip
\paragraph{Démarche de recherche}\mbox{} \\

% TODO: Modèles phénoménologique -> faire des expériences, démarche empirique basée sur l'expérience.

Afin d'identifier les variables du procédé d'injection-moulage des thermoplastiques qui sont susceptibles de contenir une information sur la qualité finale du produit, nous avons étudié la littérature.
Notre État de l'Art propose une revue des publications majeures sur le pilotage du procédé d'injection-moulage, depuis 1975 à ce jour.

Afin de répondre aux verrous qui limitent le déploiement du contrôle qualité, nous n'avons pas retenu la mesure invasive dans nos travaux.
Aussi, nous avons choisi de définir le périmètre de notre recherche au contrôle qualité non-invasif pour le procédé industriel.
Cela permet de limiter la complexité de la mise en place des dispositifs de mesure.

L'expérimentation occupe une part importante de ces travaux.
Dans un premier temps, cela nous a permis de comprendre le procédé et la pratique industrielle.
Nous avons étudié l'utilisation de moyens de mesures issus de l'imagerie non-conventionnel : scanner laser, thermographie, polarimétrie, ainsi que de mesures intégrées dans l'outillage que nous n'avons pas retenu.
Par la suite, nous avons cherché à valider par itérations successives notre dispositif de mesure.

À partir des mesures, il s'agit de concevoir un système capable de discerner les pièces de bonne qualité, des pièces de mauvaise qualité.
Nous orientons nos travaux dans l'utilisation de méthode d'apprentissage supervisé.
L'expert qualité humain transmet son savoir holistique au système qui réalise le contrôle de manière autonome.

Le développement du système de mesures s'inscrit dans une démarche de validation de l'ensemble de nos propositions.
Les modèles construits par apprentissage statistique nécessitent un jeu de données d'apprentissage.
Posséder un dispositif de mesures nous permet d'enregistrer les données requises pour l'apprentissage du modèle.
Le dispositif de mesures est conçu par une intégration verticale de l'imagerie au traitement des données.
Cela permet de maitriser l'ensemble de la chaîne d'acquisition et d'analyse des données, et de pouvoir l'optimiser dans son ensemble, pour répondre à notre problème.
La validation du système est alors réalisée en condition industrielle.

% TODO: Contributions avant Organisation?

\bigskip
\paragraph{Organisation du manuscrit}\mbox{} \\

Ce manuscrit est constitué de quatre chapitres qui peuvent être indépendamment parcourus.
Le lecteur sera orienté au gré de sa lecture, vers les sections d'autres chapitres, lorsque des informations complémentaires pourront lui être utiles.
Le premier chapitre \ref{ch:1} présente de manière détaillée (i) le contexte industriel du travail de doctorat ; (ii) les objectifs de recherche pour le déploiement du contrôle qualité dès la sortie de l'injection-moulage.
Le deuxième chapitre \ref{ch:2} a pour objectifs de (i) présenter les différentes technologies de métrologies existantes ; (ii) le choix de l'imagerie non-conventionnelle.
Le troisième chapitre \ref{ch:3} propose d'étudier les moyens de formalisation de la notion de Qualité par apprentissage et de les évaluer sur notre cas d'application industriel.
Le quatrième chapitre \ref{ch:4} détaille (i) la conception d'un nouveau dispositif de mesures \ ; (ii) son évaluation industrielle.
Enfin, en conclusion de ce manuscrit un bilan concernant les travaux de ces trois années de doctorat est dressé et une discussion est proposée concernant les nombreuses perspectives.

\bigskip
\paragraph{Contributions}\mbox{} \\

Ces travaux de thèse proposent une large étude des techniques disponibles qui sont requises pour mettre en place le contrôle automatique des produits sur les chaînes de production.
Le travail s'intéresse à l'apport de l'imagerie non-conventionnelle, aux méthodes d'apprentissages automatiques et à la conception d'un système complet de contrôle automatique de la qualité.

Ces travaux de thèse ont permis le développement d'un système de contrôle automatique de la qualité des pièces plastiques, dès la sortie du moule, de manière non-invasive, sans contact avec la pièce et avec un impact minimal sur le procédé.
Ce système associe un banc de mesures spécialement développé pour répondre aux contraintes industrielles, ainsi qu'un logiciel afin d'intégrer et de réaliser le traitement des sources d'information.
Le logiciel peut tirer partie des possibilités de calculs distribués offert par les solutions \textit{Cloud}.
Une interface homme-machine sous forme d'\textit{application web} a été développé pour permettre l'apprentissage supervisé.
Ce système a été évalué sur le procédé d'injection-plastique des thermoplastiques, sur des productions industrielles.

\bigskip
Notre système de mesures de la qualité a été lauréat du sixième concours \textit{OutOfLab}\footnote{\href{http://outoflabs.linksium.fr/resultats/}{Résultats de la sixième édition du concours \textit{OutOfLabs}}}, organisé par la \textit{SATT}\footnote{\textit{SATT} : Société d'Accélération du Transfert de Technologies} \href{https://www.linksium.fr/}{Linksium Grenoble Alpes}.
Ce prix nous permet de poursuivre la recherche et les développements vers une utilisation commerciale.

\bigskip
Ces travaux de thèse ont fait l’objet de communications en congrès scientifiques internationaux \cite{nagorny_towards_2017, nagorny_injection_2017, nagorny_quality_2017, nagorny_generative_2018, nagorny_polarimetric_2019}.
