%%%%%%%%%%%%%%%%%%%%%%%%%%%%%%%%%%%%%%%%%%%%%%%%%%%%%%%%%%%%%%%%%%%%%%%%%%
%%%%%                        Intro Générale                         %%%%%%
%%%%%%%%%%%%%%%%%%%%%%%%%%%%%%%%%%%%%%%%%%%%%%%%%%%%%%%%%%%%%%%%%%%%%%%%%%
\phantomsection 
\addcontentsline{toc}{chapter}{Introduction}
\addtocontents{toc}{\protect\addvspace{10pt}}

\vspace*{-1cm}
\begin{flushright}
\section*{\fontsize{20pt}{20pt}\selectfont\textnormal{Introduction}}
\end{flushright}
\vspace{2cm}

\lhead[\fancyplain{}{Introduction}]
      {\fancyplain{}{}}
\chead[\fancyplain{}{}]
      {\fancyplain{}{}}
\rhead[\fancyplain{}{}]
      {\fancyplain{}{Introduction}}
\lfoot[\fancyplain{}{}]%
      {\fancyplain{}{}}
\cfoot[\fancyplain{}{\thepage}]
      {\fancyplain{}{\thepage}}
\rfoot[\fancyplain{}{}]%
     {\fancyplain{}{\scriptsize}}
     

%%%%%%%%%%%%%%%%%%%%%%%%%%%%%%%%%%%%%%%%%%%%%%%%%%%%%%%%%%%%%%%%%%%%%%%%%%
%%%%%                      Start part here                          %%%%%%
%%%%%%%%%%%%%%%%%%%%%%%%%%%%%%%%%%%%%%%%%%%%%%%%%%%%%%%%%%%%%%%%%%%%%%%%%%
%\lettrine[lines=1]{L}{ }'industrie manufacturière cherche à produire en grande quantité des pièces à la qualité optimale.
%La qualité d'une pièce est l'ensemble de ses caractéristiques qui répondent à un besoin.
%Les besoins spécifiées dans les cahiers des charges concernent la fonction des pièces, mais aussi leur aspect.
%En particulier, l'aspect visuel devient pour le client un indicateur de la qualité fonctionnelle.
%Contrôler la qualité d'un produit nécessite de mesurer ces caractéristiques fonctionnelles et ces caractéristiques d'aspect.
%Notre travail propose d'utiliser des techniques d'imagerie non-conventionnelle pour contrôler l'aspect et la géométrie.
%
%De plus, dans les cahiers des charges, le niveau d'exigence en matière de production est croissant.
%L'automatisation des procédés de fabrication demande de pouvoir écarter les pièces non-conformes de la chaîne de production.
%En particulier, les assemblages automatisés de plusieurs pièces augmentent singulièrement la sensibilité aux produits non-conformes.
%En effet, la présence de produits non conformes, dans un processus automatisé, peut conduire à des arrêts de chaîne coûteux.
%Dans le cadre de l'injection-moulage des thermoplastiques, les secteurs automobile et médical sont pionniers de cette automatisation. 
%Le taux de pièces non-conformes acceptable est particulièrement faible.
%Aujourd'hui, les niveaux de qualité inférieurs à cinq pièces non-conformes par million (5 ppm) ne peuvent être garantit que par le contrôle de cent pourcent des pièces.
%
%% Il est alors indispensable de réaliser le contrôle de la qualité de cent pourcent des pièces.
%Cependant, lorsqu'il est réalisé par un opérateur humain, le coût économique du contrôle de la qualité à cent pourcent n'est pas viable pour un volume de production industriel.
%Le contrôle qualité ne doit pas non plus interrompre le flux de production.
%C'est pourquoi l'automatisation du contrôle de la qualité est obligatoire.
%Un des objectif de notre travail est de réaliser le contrôle de la qualité directement en ligne de production.
%
%% Les coûts de non-qualité sont engendrés par la fabrication de produits non-conformes.
%Il est coûteux d'apporter de la valeur ajoutée par des opérations de finition à un produit qui pourrait être déclaré non conforme dès l'étape qui a causée la non-conformité.
%Dans le cadre de l'injection-moulage des thermoplastiques, le contrôle de la qualité des produits dès la sortie du moule permet d'éliminer le coût de non-qualité.
%La gestion des produits non-conforme est également simplifiée.  % La gestion du cycle de vie 
%% Si elle est contrôlée à temps, une pièce non-conforme pourrait être recyclée dès sa production.
%% De plus, un circuit spécifique de recyclage pourrait-être mis en place à chaque étape de contrôle de la qualité, tout au long de la production.
%
%Dans une démarche de maîtrise du procédé de fabrication, le contrôle de la qualité en ligne pourrait permettre de détecter la dérive du procédé.
%Il deviendrait alors possible d'alerter ou de stopper le moyen de production, pour ne plus produire de pièces non-conformes.
%Cela permettrait d'économiser la matière première de plusieurs centaines, voir de plusieurs milliers de pièces.
%Si ces pièces poursuivent leur chemin dans des étapes de finitions, le coût de la finition est également économisé.
%Dans notre cadre d'étude, le procédé d'injection-moulage des thermoplastiques est généralement stable.
%C'est pourquoi la dérive du procédé apparait rarement.
%Un opérateur humain peut superviser généralement plusieurs machines en parallèle.
%Il est plus courant de devoir écarter des pièces non-conformes dûs à des causes exceptionnelles (défaillance du procédé), qu'à cause de la dérive du procédé.
%Dans le cas d'une défaillance exceptionnelle, le contrôle en ligne permet une réactivité maximale pour réparer et ne pas impacter l'ensemble de la chaîne de fabrication.
%Dans le cas d'une dérive du procédé, le contrôle en ligne permet d'alerter, voir d'interrompre la production.
%
%% Enfin, le contrôle qualité est une opération indispensable pour respecter le cahier des charges.
%% Cependant, cette opération n'ajoute pas de valeur au produit final.
%Enfin, le contrôle qualité n'ajoute pas de valeur au produit final.
%Aussi, le coût acceptable de l'installation d'un système de contrôle automatique est généralement faible.
%Seules les productions hautement spécialisées investissent dans son automatisation.
%Le coût du système de contrôle doit être inférieur au coût de la production de pièces non-conformes.
%Un objectif important de notre travail sera de proposer un dispositif de contrôle de la qualité peu coûteux, afin qu'il puisse être largement adopté dans l'industrie de la plasturgie.  % toute l'industrie
%
%Plusieurs contraintes technologiques entrent en jeu dans la conception d'un système prévu pour être intégré en ligne de production.
%En premier lieu, la durée du cycle du procédé de production doit être respecté.
%Le temps de cycle ne doit pas être allongé par la durée de la mesure.
%De plus, les lignes de production sont désormais en constante évolution.
%Un moyen de mesure mobile permettrait d'être positionner au grès des besoins de contrôle.
%Enfin, la notion de qualité d'une pièce doit être spécifiée au sein du système de mesure afin que le système puisse déterminer si la pièce est conforme ou non-conforme.
%Pour répondre à ces contraintes, notre travail propose d'évaluer l'intérêt de l'imagerie non-conventionnelle des pièces dès la sortie du moule.
%Notre approche du traitement des données issues de capteurs s'appuient sur l'apprentissage statistique.
%L'expert humain transmet sa connaissance de la notion de qualité au dispositif présenté. % à notre système.
%Il annote les pièces qui sont produites au fur et à mesure du réglage de la presse à injecter.
%Le système apprend alors à associer les pièces conformes ou non-conformes avec les mesures brutes des capteurs.

%  - - - - - - - - - - - - - - - - - -
% Previous removed by Eric
%  - - - - - - - - - - - - - - - - - -

% Un objectif à long terme fait suite à l'installation d'un système de contrôle de la qualité dès la sortie du moule.
% Il devient envisageable d'ajuster les réglages du procédé pièce après pièce, afin d'optimiser la qualité du produit.
% Enfin, il devient possible d'effectuer une rétro-action sur les réglages du procédé afin d'optimiser la qualité produite.

%L'injection-moulage des thermoplastiques est un procédé à haute cadence, peu coûteux car répétable.
%Le moyen de production est également amorti sur des millions de pièces.
%Une fois la machine de production réglée, le procédé est en général stable.
%Il peut produire de manière continue plusieurs milliers de pièces sans intervention humaine.
%Le coût de la matière première pour le moulage est faible en comparaison du coût des étapes de finitions suivantes, telles que les étapes de peintures.
%Éviter aux pièces de mauvaise qualité d'être peintes est une priorité.
%En second temps, un système de rétroaction pourra être déployé sur le procédé, afin de maximiser la qualité produite.
%Ce n'est pas la priorité économique.
%Le procédé d'injection-moulage des thermoplastiques consiste à injecter un polymère visqueux dans un outillage massif refroidi ; à maintenir une pression de compactage pendant une durée spécifique ; puis à éjecter la pièce tout en préparant en parallèle la matière nécessaire au cycle suivant.
%Il est possible d'ajuster précisément les pressions et les durées d'ouvertures et de fermetures de multiples buses d'injection de manière séquentiel.
%Ainsi, l'espace des variables de pilotage est grand et les variables sont continues.
%
%\noindent
%La production de pièce par moulage des thermoplastiques suit généralement la séquence suivante :
%\begin{enumerate}
%\item Injection-moulage de la pièce
%\item Stockage tampon des pièces
%\item Étapes de peintures
%\item Stockage
%\item Expédition
%\end{enumerate}
%
%Afin de minimiser les coûts d'immobilisation du stockage, on cherche à réduire au maximum le stockage entre les étapes.
%Les défauts géométriques et d'aspect apparaissent majoritairement lors des étapes de déplacements des pièces entre les étapes successives de la production.
%Pour des pièces techniques au cahier des charges rigoureux, un contrôle qualité avant expédition est donc obligatoire.
%Les défauts géométriques et d'aspect sont également produits lors de l'étape initiale d'injection-moulage.
%De nombreux défauts dit "d'aspect" sont des défauts géométriques à l'échelle de la centaine de micromètres.
%Ceux-ci modifient les réflexions des lumières incidentes et cause des changements de luminosité brusques, qui ne sont pas acceptables pour le cahier des charges du produit fini.
%Les étapes de peintures peuvent mettre en valeurs des défauts en introduisant des surfaces spéculaires.

% \bigskip
% \paragraph{Problématique de recherche}\mbox{} \\
% \paragraph{Le FUI SAPRISTI : un projet de recherche collaborative de 2016 à 2019}\mbox{} \\

% \bigskip
\lettrine[lines=1]{C}{ }e travail de doctorat s'inscrit dans le projet de recherche collaborative FUI\footnote{FUI : Fond Unique Interministériel} SAPRISTI\footnote{SAPRISTI : Système Autocorrectif pour la PRoduction zéro défaut de pIèces pLaSTIques} du \textit{21\up{ème} appel à projet FUI Régions.}\footnote{\href{https://competitivite.gouv.fr/les-appels-a-projets-de-r-d-fui/le-21e-appel-a-projets/les-resultats-du-21e-appel-a-projets-787.html}{Résultats du 21\up{ème} appel à projets FUI sur competitivite.gouv.fr} : \textit{60 projets bénéficieront d’une aide de l’État de 43 Millions d'Euros ainsi que d’une aide des collectivités territoriales et des fonds communautaires (FEDER) de 37,8 Millions d'Euros.}}.
L’obtention d’une chaîne de fabrication compétitive passe par l’optimisation du procédé de production.
Le projet FUI SAPRISTI s'intéresse en particulier au procédé d'injection-moulage des thermoplastiques dans le secteur automobile.
C'est pourquoi il est labellisé par trois pôles de compétitivité :
\begin{itemize}
	\item \href{http://www.plastipolis.fr/}{Plastipolis},
	\item \href{https://www.id4car.org}{iDforCAR},
	\item \href{https://www.montblancindustries.com/}{Mont-Blanc Industries}.
\end{itemize}
\noindent
Le consortium FUI SAPRISTI est constitué de laboratoires académiques, d'industriels transformateurs et du Centre Technique français de la plasturgie IPC.
Ainsi, les partenaires du consortium SAPRISTI sont :
\begin{itemize}
\item \href{http://www.acsysteme.com/}{Acsystème}, développeur et intégrateur de solutions en automatique et traitement de l'information,
\item Le \href{https://ct-ipc.com/}{Centre Technique IPC}, acteur européen en développement et recherche pour le secteur de la plasturgie,
\item Le \href{https://lamcos.insa-lyon.fr/}{laboratoire LaMCoS} de l'\href{https://www.insa-lyon.fr/}{INSA de Lyon}, expert de la modélisation du procédé d'injection-moulage des thermoplastiques,
\item \href{https://www.plasticomnium.com/}{Plastic Omnium}, équipementier automobile mondial
\item \href{https://www.renault.fr/}{Renault}, automobiliste concepteur et équipementier,
\item \href{http://www.sise-plastics.com/}{SISE}, acteur européen de la régulation des procédés en plasturgie, dont notamment le contrôle de l'injection séquentiel,
\item Le \href{http://www.symme.univ-smb.fr/}{laboratoire SYMME}\footnote{SYMME : SYstèmes et Matériaux pour la MÉcatronique} de l'\href{https://www.univ-smb.fr/}{Université Savoie Mont Blanc}, expert en réglage automatique et en optimisation de procédés industriels, dans le cadre de son axe de recherche sur la \href{https://symme.univ-smb.fr/axes/qualite-indusrtielle}{Qualité industrielle}.
\end{itemize}

L'injection-moulage des thermoplastiques est un procédé de production bas-coût et haut-volume.  % (\textit{Low-Cost, High Volume}).
C'est un procédé particulièrement intéressant car il permet d'obtenir avec un temps de cycles courts des pièces de toutes dimensions aux géométries complexes ; qui sont très proches du produit final.
Le marché de la plasturgie représente en France 3500 entreprises pour 30,2 Milliards de Chiffre d'Affaires en 2018\footnote{Synthèse 2018 du secteur réalisée par la Fédération Française de la Plasturgie.}.
Dans ce contexte, l’enjeu du projet FUI SAPRISTI est triple :
\begin{itemize}
	\item Déterminer des indicateurs pertinents à mettre sous contrôle.
	\item Mieux comprendre les phénomènes physiques qui entrent en jeu dans le procédé, afin de modéliser le procédé en prenant en compte les indicateurs sélectionnés.
	\item Optimiser le procédé à partir de la modélisation et des indicateurs, afin de produire une qualité optimale. % Optimiser le procédé en boucle fermée 
\end{itemize}

% Dans ce cadre, le laboratoire SYMME a des objectifs précis.
Ce travail de doctorat s'inscrit dans ce projet.
% Nous chercherons à déterminer quelles sont les variables pertinentes à étudier sur le procédé pour déterminer la qualité de la production.
% C'est pourquoi nous avons dans un premier temps chercher déterminer quelles étaient les variables pertinentes à étudier sur le procédé.
% Ces variables peuvent être des mesures sur la presse à injecter ou bien des mesures réalisées sur le produit.
% Ensuite, il s'agit de modéliser la relation entre ces variables et la qualité finale des produits.
% Pour cela, nous utiliserons des modèles construits par apprentissage statistique.
L'objectif à long terme du FUI SAPRISTI est de mettre sous contrôle le procédé, afin de maximiser la qualité de la production.

Nous avons choisi de mettre en place un moyen de contrôle des caractéristiques des pièces en ligne de production.
D'autres approches réalisent des mesurages sur le procédé plutôt que sur le produit.
% Cet objectif ne peut être atteint que si un moyen de contrôle des caractéristiques des pièces, en ligne de production, est mis en place.
C'est pourquoi nous avons orienté nos travaux vers la conception d'un moyen de mesure automatique de la qualité des produits, en ligne de production.
% Cette nécessité de disposer d'un moyen de contrôle automatique de la qualité des produits a donné lieu à notre problématique de recherche.

Dans le cas de l’injection-moulage des thermoplastiques, aucun système de mesure de la qualité ne répond actuellement aux exigences des partenaires du FUI SAPRISTI \ ; en particulier avec les contraintes techniques issues de l'intégration de la mesure de la qualité en ligne de production.
Des échanges avec les partenaires du FUI SAPRISTI nous ont permis de prendre en compte les limites que posent le procédé industriel.
Nous avons identifié trois verrous au déploiement massif du contrôle qualité à cent pourcent dès la sortie de la machine :
\begin{itemize}
	\item économique : pour être déployé sur chaque machine le coût du système doit être faible,
	\item technologique: respect de la durée du cycle, peu invasif pour le procédé de production et capable d'extraire l'information sur la qualité des produits à partir des mesures,
	\item humain : capable de modéliser la perception qualité de l'expert humain.
\end{itemize}
Nous évaluerons et chercherons à résoudre chacun de ces verrous afin de proposer un moyen de mesure industriellement viable.
% Nous présentons les méthodes et les moyens retenus pour lever ces verrous.

\bigskip
% \paragraph{Démarche de recherche}\mbox{} \\

% TODO: Modèles phénoménologique -> faire des expériences, démarche empirique basée sur l'expérience.
Afin d'identifier les variables du procédé qui sont susceptibles de contenir une information sur la qualité du produit, nous avons réalisé une étude de la littérature.
% Notre état de l'art propose une revue des publications majeures sur le pilotage du procédé d'injection-moulage, depuis 1975 à ce jour.
De nombreuses variables ont été étudiées depuis 1975.
La pratique industrielle actuelle propose d'utiliser des mesurages intégrées dans l'outillage.
L'intégration des capteurs nécessaire à ces mesures est invasive, puisqu'il est nécessaire de modifier les outillages.
C'est à dire que le travail d'ingénierie nécessaire à la conception de l'outillage en intégrant les capteurs est coûteux.
De plus, l'instrumentation des capteurs est fragile et leur maintenance nécessite un arrêt prolongé de la production, puisque c'est l'outillage complet qui doit être expertisé.
Afin de lever les verrous économiques et technologiques qui limitent le déploiement du contrôle qualité, nous n'avons pas retenu la mesure invasive dans nos travaux.
Cela permet de limiter la complexité de la mise en place des dispositifs de mesure.
C'est pourquoi nous avons défini le périmètre de notre recherche au contrôle non-invasif pour le procédé industriel, de la qualité des produits.

L'expérimentation occupe une part importante de nos travaux.
Dans un premier temps, cela nous a permis de comprendre le procédé et la pratique industrielle.
Dans un second temps, nous avons étudié l'utilisation de moyens de mesurages issus de l'imagerie non-conventionnelle : scanner laser, thermographie, polarimétrie, ainsi que de mesures intégrées dans l'outillage que nous n'avons pas retenu.
L'imagerie non-conventionnelle permet d'acquérir des informations que l'œil humain ne peut percevoir. % de l'expert
% Par la suite, nous avons cherché à valider par itérations successives notre dispositif de mesure.
% Notre démarche de recherche est phénoménologique.
Nous nous appuyons sur l'expérimentation et la mesure pour construire nos modèles.  %, plus que sur les théories chimico-thermo-dynamique des polymères.
Dans le cas où l'on dispose d'un grand nombre de cas d'étude, cette démarche est particulièrement adaptée à l'utilisation de l'apprentissage statistique pour modéliser les phénomènes.

À partir des résultats des mesures, il s'agit de concevoir un système capable de discerner les pièces conformes, des pièces non-conformes.
Nous orientons nos travaux dans l'utilisation de méthode d'apprentissage statistique.
L'expert qualité humain transmet son savoir au système de mesure, qui réalise le contrôle de manière autonome.

\bigskip

Ces travaux de doctorat ont fait l’objet de communications en congrès scientifiques internationaux \cite{nagorny_towards_2017, nagorny_injection_2017, nagorny_quality_2017, nagorny_generative_2018, nagorny_polarimetric_2019}, ainsi que d'un colloque national \cite{nagorny_towards_2017}.

\bigskip
% \paragraph{Organisation du manuscrit}\mbox{} \\

Ce manuscrit est constitué de quatre chapitres qui peuvent être indépendamment parcourus.
Le lecteur sera orienté au gré de sa lecture, vers les sections d'autres chapitres, lorsque des informations complémentaires pourront lui être utiles.

Le premier chapitre présente de manière détaillée le contexte industriel du travail de doctorat et les objectifs de recherche pour le déploiement du contrôle qualité dès la sortie de l'injection-moulage.  % \ref{ch:objectives}  (i) le contexte industriel ; (ii) les objectifs
Le second chapitre a pour objectifs de présenter les différentes technologies de métrologies existantes et de justifier notre choix d'utilisation de l'imagerie non-conventionnelle thermographique et polarimétrique à partir d'un essai expérimental.  %  \ref{ch:measure} (i) présenter ; (ii) discuter
Le troisième chapitre propose d'étudier les méthodes de formalisation de la notion de qualité par apprentissage et de les évaluer sur notre cas d'application industriel.
Nous présenterons les méthodes d'apprentissage supervisé et non-supervisé, ainsi que la démarche d'annotation d'un jeu de données d'apprentissage.  % \ref{ch:metric_learning} 
Enfin, le quatrième chapitre détaille la conception d'un nouveau dispositif de mesure et réalise son évaluation sur deux cas d'applications industrielles.  % \ref{ch:theeye} (i) la conception ; (ii) son évaluation

En conclusion, un bilan concernant les travaux de ces trois années de doctorat sera dressé.
Une discussion sera proposée concernant les nombreuses perspectives de recherche que nous avons identifiées.  % \ref{ch:conclusion}
