%%%%%%%%%%%%%%%%%%%%%%%%%%%%%%%%%%%%%%%%%%%%%%%%%%%%%%%%%%%%%%%%%%%%%%%%%%
%%%%%                   REMERCIEMENTS & DEDICACE                    %%%%%%
%%%%%%%%%%%%%%%%%%%%%%%%%%%%%%%%%%%%%%%%%%%%%%%%%%%%%%%%%%%%%%%%%%%%%%%%%%
\phantomsection 
\addcontentsline{toc}{section}{Remerciements \& Dédicace}
\addtocontents{toc}{\protect\addvspace{5pt}}

\lhead[\fancyplain{}{Remerciements \& Dédicace}]
{\fancyplain{}{}}
\chead[\fancyplain{}{}]
{\fancyplain{}{}}
\rhead[\fancyplain{}{}]
{\fancyplain{}{Remerciements \& Dédicace}}
\lfoot[\fancyplain{}{}]
{\fancyplain{}{}}
\cfoot[\fancyplain{}{\thepage}]
{\fancyplain{}{\thepage}}
\rfoot[\fancyplain{}{}]%
{\fancyplain{}{\scriptsize}}

\vspace*{0mm}
\begin{flushright}
	\fontsize{20pt}{20pt}\selectfont\textnormal{Remerciements}
\end{flushright}

Ces quelques lignes seront insuffisantes pour remercier tous ceux qui doivent l'être.
% Je m'excuse par avance de ne pas avoir nominativement cité les nombreuses personnes concernées.
Le lecteur se devra d'effectuer une recherche bibliographique pour réunir la liste complète de ces personnes : co-auteurs, membres de groupements de travaux, doctorants de l'École Doctorale SISEO en 2017-2020, les étudiants de l'IUT d'Annecy et de l'école d'ingénieur Polytech Annecy qui ont été d'excellents cobayes.

\bigskip
\noindent
Je souhaite remercier les professeurs Christophe Cudel et Fabrice Mériaudeau pour l'intérêt qu'ils ont porté à ce travail et pour leurs commentaires détaillés qui ont permis d'améliorer ce manuscrit.
Merci au professeur Gilles Régnier d'avoir accepter de présider le jury ; ainsi qu'à messieurs Ronan Le Goff, Émilio Vitale pour leurs présences le jour J et leurs aides pour la réalisation des essais expérimentaux industriels,  indispensable pour avancer dans mon travail.

\bigskip
\noindent
Je remercie particulièrement Éric Pairel et Maurice Pillet pour :
\begin{itemize}
\item la direction idéale de ces travaux de recherche,
\item leur soutien humain infaillible,
\item le partage de leurs expériences académique, professorale et de recherche.
\end{itemize}
Le financement de ce travail et son accomplissement, n'aurait pû être obtenu sans leur persévérance.
\noindent
Je remercie BPI France et de manière générale l'effort de la France pour le financement de la Recherche.
Merci aux Bibliothèques Universitaires pour leur travail de préservation des ouvrages de références, et à tous les sites de l'Internet qui rendent accessibles une vaste connaissance librement, \textit{HAL}, \textit{arXiV}, \textit{Science-Hub} qui ont rendus possible l'étude bibliographique.

\bigskip
\noindent
Cette ligne me permet de remercier efficacement l'ensemble des membres du laboratoire SYMME d'Annecy\textendash Le Bourget-du-Lac pour la bonne ambiance qu'ils entretiennent.
Je garde un excellent souvenir de tous les fructueux échanges qui m'ont permis d'approfondir ou de faire germer des idées.
Merci aux chercheurs, ingénieurs, administratifs et aux doctorants pour avoir égayés tous ces moments ; en particulier mon voisin de bureau Orlando ; le couloir du haut : Amandine, Florian, Jérémy, Koki, Luc, Marine, Mathias, Mickaël, Thomas, Thomas, Thomas ; le couloir du bas : Amélie, Aurélien, Christian, Ludovic, Mathias, Thibault ; les voisines de l'IREGE / label RES : Camille, Caroline, Olga ; les voisins du Bourget, Jean, Marie ; et pour la fabrication additive Blaise et Hugues.

\bigskip
\noindent
Coté famille, je dois d'énormes mercis à mes grands-parents, mes parents modèles \textemdash Monique et Fred\textemdash, la merveilleuse Margot et un joyeux enfant qui a égayé la rédaction de ce manuscrit.

\bigskip
\noindent
Enfin, merci à vous, lecteurs, qui faites vivre ce manuscrit.
J'espère qu'il vous permettra d'obtenir des réponses ou qu'il saura créer de nombreuses interrogations, tout en vous divertissant, avant de vous endormir.

\vspace*{5mm}
\begin{flushright}
	\fontsize{20pt}{20pt}\selectfont\textnormal{Dédicace}
\end{flushright}
\begin{flushright}
	\emph{À ma généreuse famille, pour tout le soutien reçu, qui m'a permis de mener à bien ce travail.}
\end{flushright}

\newpage  % If it's not here then the page header is not Remerciements Dédicaces
