%%%%%%%%%%%%%%%%%%%%%%%%%%%%%%%%%%%%%%%%%%%%%%%%%%%%%%%%%%%%%%%%%%%%%%%%%%
%%%%%                   REMERCIEMENTS & DEDICACE                    %%%%%%
%%%%%%%%%%%%%%%%%%%%%%%%%%%%%%%%%%%%%%%%%%%%%%%%%%%%%%%%%%%%%%%%%%%%%%%%%%
\phantomsection 
\addcontentsline{toc}{section}{Remerciements \& Dédicace}
\addtocontents{toc}{\protect\addvspace{5pt}}

\lhead[\fancyplain{}{Remerciements \& Dédicace}]
{\fancyplain{}{}}
\chead[\fancyplain{}{}]
{\fancyplain{}{}}
\rhead[\fancyplain{}{}]
{\fancyplain{}{Remerciements \& Dédicace}}
\lfoot[\fancyplain{}{}]
{\fancyplain{}{}}
\cfoot[\fancyplain{}{\thepage}]
{\fancyplain{}{\thepage}}
\rfoot[\fancyplain{}{}]%
{\fancyplain{}{\scriptsize}}

\vspace*{-1cm}
\begin{flushright}
	\section*{\fontsize{20pt}{20pt}\selectfont\textnormal{Remerciements}}
\end{flushright}

Ces quelques lignes seront nettement insuffisantes pour remercier ceux qui doivent l'être.
Je m'excuse par avance de ne pas avoir nominativement cité les nombreuses personnes concernées.
Le lecteur se devra d'effectuer une recherche bibliographique pour réunir la liste complète de ces personnes : co-auteurs, membres de groupements de travaux, doctorants de l'École Doctorale SISEO en 2016-2019 et les étudiants de l'école d'ingénieur Polytech Annecy qui ont été d'excellents cobayes d'enseignement.

\bigskip
\noindent
Je souhaite remercier les membres du jury M. XXX et M. XXX pour l'intérêt qu'ils ont porté à ce travail et à M. XXX pour avoir été le rapporteur.
Leurs remarques et suggestions furent précieuses.

\bigskip
\noindent
Je remercie particulièrement Éric Pairel et Maurice Pillet, mes directeurs de thèse pour :
\begin{itemize}
\item leur soutien humain infaillible,
\item la direction idéale de ces travaux de recherche,
\item le partage de leurs expériences académique, professorale, de recherche.
\end{itemize}
Le financement de ce travail et son accomplissement, n'aurait pû être obtenu sans leur persévérance.
\noindent
Je remercie BPI France d'avoir financé ce travail, et de manière générale l'effort de la France pour le financement de la Recherche.
Merci aux Bibliothèques Universitaires pour leur travail de préservation des ouvrages de références, et à tous les sites de l'Internet qui rendent accessibles une si vaste connaissance.

\bigskip
\noindent
Cette ligne me permet de remercier l'ensemble des membres du laboratoire SYMME d'Annecy\textendash Le Bourget-du-Lac.
Je garde un excellent souvenir de tous les fructueux échanges qui m'ont permis d'approfondir et de faire germer les idées.
Merci aux chercheurs, ingénieurs et aux doctorants pour avoir égayés tous ces moments.
% ; mon voisin de bureau Orlando ; le couloir du haut : Thomas, Amandine, Jérémy, Mathias, Thomas, Thomas, Florian, Luc ; le couloir du bas : Aurélien, Thibault, Christian, Mathias, Ludovic ; les voisines du label RES : Caroline, Camille, Olga, et les voisins du Bourget, dont Jean.

\bigskip
\noindent
Coté famille, je dois d'énormes mercis à mes grands-parents \textemdash Mamé\textemdash, mes parents modèles \textemdash Maman Monique et Papa Fred\textemdash, ma merveilleuse femme Margot et nos joyeux enfants qui ont égayés la rédaction de ce manuscrit.
Merci Margot pour tes relectures attentives qui ont permis de corriger ce manuscrit.

\bigskip
\noindent
Enfin, merci à vous, lecteurs, qui faites vivre ce manuscrit.
J'espère qu'il vous permettra d'obtenir des réponses et qu'il saura créer de nouvelles interrogations, tout en vous divertissant avant de vous endormir.


\vspace*{1.5cm}
\begin{flushright}
	\section*{\fontsize{20pt}{20pt}\selectfont\textnormal{Dédicace}}
\end{flushright}
\begin{flushright}
\emph{À ma généreuse famille, pour tout le soutien reçu, qui m'a permis de mener à bien ce travail.} \\
\end{flushright}
