%%%%%%%%%%%%%%%%%%%%%%%%%%%%%%%%%%%%%%%%%%%%%%%%%%%%%%%%%%%%%%%%%%%%%%%%%%
%%%%%                            Résumé                             %%%%%%
%%%%%%%%%%%%%%%%%%%%%%%%%%%%%%%%%%%%%%%%%%%%%%%%%%%%%%%%%%%%%%%%%%%%%%%%%%
\phantomsection 
\addcontentsline{toc}{section}{Résumé}
\addtocontents{toc}{\protect\addvspace{5pt}}

\vspace*{-1cm}
\begin{flushright}
\section*{\fontsize{20pt}{20pt}\selectfont\textnormal{Résumé}}
\end{flushright}

\fancyhf{}
%\lhead[\fancyplain{}{Notations}]
%{\fancyplain{}{}}
%\chead[\fancyplain{}{}]
%{\fancyplain{}{}}
%\rhead[\fancyplain{}{}]
%{\fancyplain{}{Notations}}
\lfoot[\fancyplain{}{}]
{\fancyplain{}{}}
\cfoot[\fancyplain{}{\thepage}]
{\fancyplain{}{\thepage}}
\rfoot[\fancyplain{}{}]
{\fancyplain{}{\scriptsize}}

Le procédé d'injection moulage des thermoplastiques permet la production industrielle de produits, dont les exigences de qualités géométriques et d'aspects sont croissantes. Maîtriser la qualité du produit nécessite de maîtriser le procédé d'injection. Les contraintes posées par le respect du cycle industriel sont nombreuses. Ces travaux de doctorats s'inscrivent dans la thématique du contrôle des produits, spécifiquement intégré sur la ligne de production. Les technologies de mesures non invasives permettent le respect des contraintes industrielles, ainsi qu'une simplification de la mise en place et une mobilité dans l'atelier. L'apport de l'imagerie non-conventionnelle, avec la thermographie et la polarimétrie, est étudiée. La classification automatique des pièces selon leur qualité est effectuée à l'aide d'un modèle de la qualité de la pièce construit par apprentissage supervisé. Un dispositif de contrôle de la qualité des produits, dès la sortie du moule, est présenté. Ce dernier utilise l'information issue de multiples capteurs et l'apprentissage supervisé itératif du modèle, au fil de la production. Plusieurs cas d'études industriels permettent d'évaluer les performances et la viabilité de la solution retenue.

\vspace*{3cm}
\begin{flushright}
	\fontsize{20pt}{20pt}\selectfont\textnormal{Abstract}
\end{flushright}

The injection molding process of thermoplastics allows the industrial production of products, whose requirements of geometric qualities and aspects are increasing. Mastering the quality of the product requires mastering the injection process. The constraints posed by the respect of the industrial cycle are numerous. This doctoral work is part of the product control theme, specifically integrated into the production line. Non-invasive measurement technologies allow compliance with industrial constraints, simplification of implementation and mobility in the workshop. The contribution of non-conventional imaging, with thermography and polarimetry, is studied. The automatic classification of parts according to their quality is carried out using a model of the quality of the part built by supervised learning. A device for checking the quality of the products, right out of the mold, is presented. The latter uses information from multiple sensors and the iterative supervised learning of the model, during production, during the process adjustment phase. Several cases of industrial studies make it possible to evaluate the performance and the viability of the chosen solution.
