% %%%%%%%%%%%%%%%%%%%%%%%%%%%%%%%%%%%%%%%%%%%%%%%%%%%%%%%%%%%%%%%%%%%%%%%%%%
% %                            PAQUETS USUELS                              %
% %                                                   %
% % How to install most of them:
% %   tlmgr install texlive-latex-extra
% % Complementary need package from texlive-2019:
% %   mathpazo courier palatino collectbox ucs
% %   adjustbox normalem ulem enumitem esvect SIstyle SIunits
% %   fundus-calligra pgf-blur animate media9 ocgx2 blindtext symbol
% %   quotchap lettrine minifp minitoc wrapfig multirow diagbox
% %   pict2e makecell placeins helvetic avantgar bbm-macros
% %   tikzsymbols algorithm2e ifoddpage relsize appendixnumberbeamer
% %   calligra
% %%%%%%%%%%%%%%%%%%%%%%%%%%%%%%%%%%%%%%%%%%%%%%%%%%%%%%%%%%%%%%%%%%%%%%%%%%

\usepackage{mathptmx}
\usepackage[T1]{fontenc}
\usepackage[utf8]{inputenc}
%\usepackage[francais]{babel}
\usepackage[french]{babel}  % francais = french babel
\usepackage{meta-donnees} % page de garde
\usepackage{meta-donnees2} %page de 4ième de couv
\usepackage{amssymb}
\usepackage{verbatim}
\usepackage{array}	
\usepackage{color}
\usepackage{cite}
\usepackage{xfrac}
\usepackage{enumitem}
\usepackage{esvect}
\usepackage[squaren,Gray]{SIunits}
\usepackage{sistyle}
\usepackage{eurosym}  %pour obtenir le symbole Euro
%\usepackage{gensymb} %pour obtenir le symbole \degree
\usepackage{calligra}
\usepackage{tikz}
\usetikzlibrary{positioning,backgrounds,fadings,shadows.blur,shadows}
\usetikzlibrary{fit,shapes.misc,shapes}

%\usepackage{animate}
\usepackage{blindtext}

%%%%%%%%%%%%%%%%%%%%%%%%%%%%%%%%%%%%%%%%%%%%%%%%%%%%%%%%%%%%%%%%%%%%%%%%%%
%%%%%                   Notations personnalisées                    %%%%%%
%%%%%%%%%%%%%%%%%%%%%%%%%%%%%%%%%%%%%%%%%%%%%%%%%%%%%%%%%%%%%%%%%%%%%%%%%%

% \newcommand{\bigO}{\mathcal{O}} % BigO notation
\newcommand{\bigO}{\operatorname{O}} % BigO notation
% \newcommand{\bigO}{O} % BigO notation


%%%%%%%%%%%%%%%%%%%%%%%%%%%%%%%%%%%%%%%%%%%%%%%%%%%%%%%%%%%%%%%%%%%%%%%%%%
%%%%%           Packages pour les entetes et pied de page           %%%%%%
%%%%%%%%%%%%%%%%%%%%%%%%%%%%%%%%%%%%%%%%%%%%%%%%%%%%%%%%%%%%%%%%%%%%%%%%%%

%\usepackage{picins}
\usepackage{fancyhdr}
%\usepackage{psboxit}  %%% A Inserer avant babel !!!! 
\usepackage{pifont}


%%%%%%%%%%%%%%%%%%%%%%%%%%%%%%%%%%%%%%%%%%%%%%%%%%%%%%%%%%%%%%%%%%%%%%%%%%
%%%%%                   Packages et couleurs perso                  %%%%%%
%%%%%%%%%%%%%%%%%%%%%%%%%%%%%%%%%%%%%%%%%%%%%%%%%%%%%%%%%%%%%%%%%%%%%%%%%%

\usepackage{xcolor}  %%% Incompatibilité avec \usepackage{colortbl} ??????
\definecolor{BleuCyan}{RGB}{0,190,190}  %%% Définition d'une couleur personnelle
\definecolor{RoseRose}{RGB}{238,44,44}
\definecolor{VertVert}{RGB}{10,255,118}
\definecolor{Anthracite}{RGB}{91,124,151}
\definecolor{GrisPasTropClair}{RGB}{83,135,135}
\definecolor{BleuPetrole}{RGB}{0, 0 ,205}
\definecolor{BleuClair}{RGB}{234, 255 ,255}
\definecolor{Bleu1}{RGB}{26, 64 ,145}
\definecolor{Rouge1}{RGB}{215, 19 ,24}
\definecolor{myblue}{rgb}{.8, .8, 1}
\definecolor{violet1}{rgb}{0.78,0.53,0.97}
\definecolor{myblue2}{rgb}{0,0.41,0.54} % dark blue
\definecolor{myred}{RGB}{192,0,0} % dark red
\definecolor{mygreen2}{RGB}{0,120,0} % dark green
\definecolor{myblue3}{HTML}{1a4091} % dark blue 	
\definecolor{couleur_marie}{RGB}{0,102,204} % dark green
\definecolor{couleur_pierre}{RGB}{6,69,173} % dark green

%\newcommand*\maboite[1]{%
%\fcolorbox{GrisPasTropClair}{BleuClair}{\hspace{1em}#1\hspace{1em}}}
%\newcommand{\parttoccolor}{blue}
%\newcommand{\chaptertoccolor}{red}
%\newcommand{\sectiontoccolor}{green!70!black}



%%%%%%%%%%%%%%%%%%%%%%%%%%%%%%%%%%%%%%%%%%%%%%%%%%%%%%%%%%%%%%%%%%%%%%%%%%
%%%%%            Packages pour les captions optimisés               %%%%%%
%%%%%%%%%%%%%%%%%%%%%%%%%%%%%%%%%%%%%%%%%%%%%%%%%%%%%%%%%%%%%%%%%%%%%%%%%%
\usepackage[font=small,font={it}]{caption} % \usepackage[small,hang]{caption2} apparement cpation2 est obsolette
%\captionsetup[table]{position=bottom}
%\renewcommand{\captionfont}{\it \small}
%\renewcommand{\captionlabelfont}{\it \bf \small}
% \renewcommand{\captionlabeldelim}{ :}  % Ne marche qu'avec caption2



%%%%%%%%%%%%%%%%%%%%%%%%%%%%%%%%%%%%%%%%%%%%%%%%%%%%%%%%%%%%%%%%%%%%%%%%%%
%%%%%           Packages pour les titres de chapitres               %%%%%%
%%%%%%%%%%%%%%%%%%%%%%%%%%%%%%%%%%%%%%%%%%%%%%%%%%%%%%%%%%%%%%%%%%%%%%%%%%
%\usepackage[Bjornstrup]{fncychap}
\usepackage[avantgarde]{quotchap}

\usepackage{lettrine}    %Lettrine exemple: \lettrine[lines=2]{L}{orem ipsum}
\usepackage[francais]{minitoc}		% Pour ajouter une table des matières à chaque chapitre
\setcounter{minitocdepth}{2}

\usepackage{hyphenat}
% \DeclareCaptionTextFormat{new}{\nohyphens{#1}}
% \usepackage{titlesec}
% \titleformat{\section}[<shape>]{<format>}{<label>}{<sep>}{<before-code>}[<after-code>]
% \titleformat{\section}{}{}{}{\begin{raggedright}}[\end{raggedright}]

% \renewcommand{\@seccntformat}[1]{\raggedright#1}
% \renewcommand{\arabic{section}-}
% \sectionfont{\raggedright\large\sffamily\bfseries}
% \titleformat{\section}{\raggedright\large\sffamily\bf}{\thesection}{1em}{}

%https://tex.stackexchange.com/questions/35686/avoid-hyphenation-in-chapter-title
% \usepackage{etoolbox}% http://ctan.org/pkg/etoolbox
% \patchcmd{\@maketitle}{#1}{\hyphenpenalty=10000 #1}{}{}% Patch \title
% \begin{raggedright}
% \end{raggedright}
% \patchcmd{\@Soustitre}{#1}{\hyphenpenalty=10000 #1}{}{}% Patch \ThesisSubTitle
% \patchcmd{\@ThesisSubTitle}{#1}{\hyphenpenalty=10000 #1}{}{}% Patch \ThesisSubTitle
% \patchcmd{\@makechapterhead}{#1}{\hyphenpenalty=10000 #1}{}{}% Patch \chapter
% \patchcmd{\@makeschapterhead}{#1}{\hyphenpenalty=10000 #1}{}{}% Patch \chapter*
% \makeatother

% https://tex.stackexchange.com/questions/184280/section-headings-with-long-titles
\pretolerance=10000 

%%%%%%%%%%%%%%%%%%%%%%%%%%%%%%%%%%%%%%%%%%%%%%%%%%%%%%%%%%%%%%%%%%%%%%%%%%
%%%%%                Packages pour les figures                      %%%%%%
%%%%%%%%%%%%%%%%%%%%%%%%%%%%%%%%%%%%%%%%%%%%%%%%%%%%%%%%%%%%%%%%%%%%%%%%%%
\usepackage{epsfig}
\usepackage{wrapfig}  %%% Inserer du texte a droite ou à gauche de l'image
%\usepackage{picins}  %%% Inserer du texte a droite ou à gauche de l'image
\usepackage{float}
\usepackage{graphicx}
%\usepackage{subfigure}
\usepackage{subcaption}
\usepackage{placeins}
\usepackage{media9}

\usepackage{fancybox}
\usetikzlibrary{arrows.meta, arrows}
\usetikzlibrary{matrix,decorations.pathreplacing}
\usepackage{adjustbox}
\usepackage{pgfplots}
\pgfplotsset{compat=1.15}
% \pgfplotsset{compat=1.3} % also works for me
% \pgfplotsset{compat=1.8} % also works for me`
% \usepackage[edges]{forest}
\usepackage{forest}
\usepackage{ccicons}

%%%%%%%%%%%%%%%%%%%%%%%%%%%%%%%%%%%%%%%%%%%%%%%%%%%%%%%%%%%%%%%%%%%%%%%%%%
%%%%%                  Packages pour les tableaux                   %%%%%%
%%%%%%%%%%%%%%%%%%%%%%%%%%%%%%%%%%%%%%%%%%%%%%%%%%%%%%%%%%%%%%%%%%%%%%%%%%
\usepackage{array}
\usepackage{textcomp}
\usepackage{booktabs}
\usepackage{colortbl}  %%% Couleurs de cellule de tableau
\usepackage{longtable}
\usepackage{lscape}    %%% Rotation des tableaux
\usepackage{makecell}
\usepackage{multirow, hhline}
\usepackage{tabularx}
\usepackage{diagbox}  %%% slashbox is supersed by diagbox with backward comp
\usepackage{pifont}
%\usepackage[table]{xcolor}

\usepackage{colortbl}

\arrayrulecolor{myblue3}
\let\oldtabular=\tabular 
\def\tabular{\small\oldtabular}

%%%%%%%%%%%%%%%%%%%%%%%%%%%%%%%%%%%%%%%%%%%%%%%%%%%%%%%%%%%%%%%%%%%%%%%%%%
%%%%%                  Packages pour les maths                  %%%%%%
%%%%%%%%%%%%%%%%%%%%%%%%%%%%%%%%%%%%%%%%%%%%%%%%%%%%%%%%%%%%%%%%%%%%%%%%%%
\usepackage{amsmath}
\usepackage{bm}

%\usepackage{empheq} % Pour encadrer les equations
%pour utiliser plusieurs fichiers bibteX
%\usepackage{biblist}



%\usepackage{fourier}
\usepackage{SIunits}



%%%%%%%%%%%%%%%%%%%%%%%%%%%%%%%%%%%%%%%%%%%%%%%%%%%%%%%%%%%%%%%%%%%%%%%%%%
%%%%%              Packages pour les liens hypertexte               %%%%%%
%%%%%%%%%%%%%%%%%%%%%%%%%%%%%%%%%%%%%%%%%%%%%%%%%%%%%%%%%%%%%%%%%%%%%%%%%%
\usepackage[backref=page]{hyperref} %%%  Packages pour les liens hypertexte a mettre après tous les autres packages
\hypersetup{
 % backref=true,
 % pagebackref=true,
 colorlinks=true,
 breaklinks=true, % permet le retour  la ligne dans les liens trop longs
 % linkcolor=BleuCyan,
 % linkcolor=couleur_marie,
 linkcolor=couleur_pierre,
 citecolor=BleuPetrole,
 filecolor=VertVert,
 % hyperindex=true,
 % bookmarks=true, %cre des signets pour Acrobat
 bookmarksopen=true,
 pdftitle={Thèse de Doctorat}, %informations apparaissant dans
 pdfauthor={Pierre Nagorny}, %dans les informations du document
 pdfsubject={Contrôle Qualité} %sous Acrobat.
 }

%\usepackage[backref=page]{hyperref}
%\renewcommand*{\backref}[1]{}
%\renewcommand*{\backrefalt}[4]{%
%	\ifcase #1 (Non cité.)%
%	\or        (Cité page~#2.)%
%	\else      (Cité pages~#2.)%
%	\fi}
